\documentclass[a4paper, 11pt]{article}


\usepackage{graphics}
\usepackage{apacite}
\usepackage{fancyhdr}


\newcommand\customfont[1]{{\usefont{T1}{custom}{m}{n} #1 }}

 \pagestyle{fancy}
 \fancyhf{}
\rhead{HR Techdoc 2017 }
\cfoot{Page \thepage}

\begin{document}
\author{Adam Bouaiss}
\customfont{
\title{Research paper} 
\maketitle
\center{\normalsize Business Skill Writing}

 \center{\includegraphics{image.png}}
 \vspace*{\fill}
 
 \hfill \textbf{Adam Bouaiss} \\

 \hfill StudentNR: 0910047

 \hfill Course Code: TINBSK0113} \\
 \thispagestyle{empty}
 \clearpage
 
 \begin{center}
 \thispagestyle{empty}
 \tableofcontents
 \clearpage
 \makeatletter
 \renewcommand{\@seccntformat}[1]{}
 \makeatother
 \setcounter{page}{1}


\section{Problem of paperwork} 
Caring for the elderly can be very demanding task for the caregivers. As if that
was not enough, filling in paperwork is also one of their duties. A survey shows
that caregivers lose 25\% of their time on paperwork. (Hanekamp, 2016) That is
a lot of time for such a simple task.
Research shows that the percentage of elderly in the dutch population will keep
on increasing. \cite{getzen1992population} Some of them will live in a home for the elderly.
This also means that more caregivers will be needed so costs will increase. It is
thus crucial that the caregivers make use of their time as efficient as possible.
A survey has found that administrative chores in long-term care cost 5 billion a
year. \cite{hanekamp2016}
Paperwork costs a caregiver more time than one would expect. Almost
everything is done on paper. Simple daily tasks such as signing medication can
result into heaps of paper after a year. It can also require a significant effort to
keep the huge amounts of papers organized.
Technology should therefore be deployed to assist them in carrying out their
work. \cite{cherry2007perceptions} This can be done by introducing a digital solution to make
filling in paperwork easier. This will reduce the time caregivers lose to filling in
paperwork and increase the time they have to do their main duty, giving care.
This is especially beneficial for the elderly as the caregivers have more time to
do their work.
Paperwork can easily be replaced by a web solution that can be accessed from
the mobile of the caregiver or using a computer. The caregiver can use this
portal to do his paperwork online. As it can be accessed using a mobile device
the caregiver can do paperwork anywhere anytime. The caregiver will no
longer have to waste time walking to the file storage and placing the paper
where it belongs. Instead he/she marks off the medication on the spot using a
smartphone and continues with his/her work.
This would significantly reduce the time a caregiver has to spend on
paperwork.

\vspace*{1mm}
\section{Solution}
A solution for paperwork would be a modular web platform. Each modules
would be designed for a specific task. For example, a module designed to be
used for anything related to the medicine a patient has to take. Caregivers can
use it to mark off what medication a patient has taken with one click. If a
patient changes medication the caregiver can easily modify it in the module,
the module then updates everything accordingly.
This would be much more flexible than doing everything on paper. A change in
medication would require a lot of papers to be re-printed and it would demand
some time for the administration to be updated accordingly. This would also
reduce the burden of paperwork for the caregiver as he can access the module
anywhere using his smartphone. \cite{cherry2007perceptions} Marking off what medication a
patient has taken would require just a click, instead of manually filling in a
paper and placing it in the file storage.
What makes this solution so innovative is that it consists of separate modules
each designed for a specific task. This solution is new and does not exist yet,
therefore not much research has been done on this subject. Normally, it would
require separate platforms, each with their own uses. However, that can easily
result in a mess as each platform has to be maintained and updated separately.
This solution, on the other hand, would be a unified platform with separate
modules each designed for a specific task. The unified platform would have a
global patient list which each module has access to, so global changes are
effective on all modules.
Each module has its own storage and works thus separate from other modules.
This has been done to ensure that if one module has a security leak, the other
modules remain inaccessible. The platform must also run on a corporate
network, which only employees have access to. The network must be not be
connected to the internet to ensure security. This means that, in order to
access the platform, the employees must be present within the facility. Doing
paperwork from home is currently no option. A possible option would be an
offline app that would let an employee do his work without being connected to
the corporate network. His work will then be automatically uploaded to the
platform the next time he connects to.
There are existing solutions for the digitization of paperwork; however each
solution is designed to solve a specific problem. If someone wanted to use
multiple platforms it would result into increased complexity and costs as each
platform needs to be set up and maintained separately.

\subsection{Benefits and disadvantages}
This solution has many benefits. First of all, it will result in a significant
reduction of paper usage, as most paperwork will be done online. Paperwork
will be done more efficiently as the caregiver can fill in paperwork digitally
whenever and wherever he wants. This will thus result into more efficient usage
of time. More efficient usage of time means less employees are needed. Less
employees means reduced costs.
Furthermore, information will be much more organised, as everything can be
looked up on the online platform.
As a lot of information, that is normally written on paper, is digitized, it
becomes much more accessible and that opens the door for data analysis.
Research has found that, by digitizing medical info of a patient, a hospital was
able to determine with more precision the optimal duration of a hospitalization.
This was done by analyzing data from previous hospital visits. In turn, this
improved recovery rates. \cite{chaudhry2006systematic}
This solution also has its downsides. First of all, it will require a big Investment
for this platform to made and set up. It will also take some time before it is up
and running, as patient info has to be added manually. The caregivers have to
be given a course on how to use the new system and it can take some time
before they get the familiar with it. And last but not least, the platform will
require to be maintained.

\section{Conclusion}
Caregivers have a crucial role in providing care for the elders that need it. If
the elderly do not get the care and attention they need, because the caregivers
are too busy with administrative tasks, that would be a serious problem that
has to be fixed. Caregivers should therefore spend as much time as possible
giving care and as few as possible doing tasks such as paperwork. Solutions
that can reduce the administrative burden caregivers have should therefore be
seriously considered.
A web platform that will digitize paperwork could be a solution. As everything is
digital, a caregiver will be able to do his administrative tasks wherever suits
him most. Everything is stored digitally and can be looked up digitally.
Caregivers will no longer have to walk to the administration office to grab a file,
but instead they can look it up on the digital platform using their smartphone.
Making changes will not result into re-printing anymore either, as all changes
done digitally are effective immediately.
Nothing comes without price, this solution will require an investment and staff
training, however in the long-term it will prove more sustainable than
paperwork.


\clearpage
\vspace*{1mm}
\bibliographystyle{apacite}
\bibliography{bibliografie}
\printindex{}

\end{center}
\end{document}
